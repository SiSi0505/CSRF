\section{Template}

\begin{table}[h]
    \centering
    \caption{Comparison of Key Parameters for MCP6001, TL084, and LM324}
    \begin{tabularx}{\textwidth}{>{\centering\arraybackslash}X >{\centering\arraybackslash}X >{\centering\arraybackslash}X >{\centering\arraybackslash}X}
        \toprule
        \textbf{Characteristic} & \textbf{MCP6001} & \textbf{TL084} & \textbf{LM324} \\
        \midrule
        Max Output (V) & $(V+)-25mV$ & $(V+)$ & $(V+) - 1.5V$ \\
        \midrule
        Min Output (V) & $(V-)+25mV$ & $(V-) + 1.5V$ & $(V-) + 150mV$ \\
        \midrule
        Supply Range (V) & $1.8V$ to $6.0V$ & $4.5V$ to $40V$ & $3V$ to $36V$ \\
        \midrule
        Offset (mV) & $\pm 4.5mV$ & $1mV$ & $\pm 3 mV$ \\
        \midrule
        Max Output (mA) & $\pm 23 mA$ $(5.5V)$ & $10 mA$ & $\pm 30 mA$ \\
        \midrule
        Idle Consumption & $100 \mu A $ & $1.4$ $mA$ & $240 \mu A$ \\
        \bottomrule
    \end{tabularx}
    \label{tab:opamp}
\end{table}
%TODO FOTOS LADO A LADO
\begin{figure}[H]
    \centering
    \includegraphics*[scale = 0.05]{Images/NovaFctHor.png}
    \caption{Logo da Nova FCT}
    \label{wrap-fig:1}
\end{figure}

%Equation System
\begin{equation}
    \begin{cases}
    
        R( 283,15 ) = 1,998\cdot 10^4 ~\Omega \\
        R( 298,15 ) = 10^4 ~\Omega\\
        R( 313,15 ) = 0,5282 \cdot 10^4 ~\Omega\\
    
    \end{cases}
    \Leftrightarrow
    \begin{cases}
        A = 1,3092 \cdot 10^{-3}\\
        B = 2,1439 \cdot 10^{-4}\\
        C = 9,6600 \cdot 10^{-8}\\
    
    \end{cases}
\end{equation}

\begin{lstlisting}[language=Matlab, caption=Matlab code example]
    Fdz
    printf('Polos: ');
    PlFdz
    %figure(3);
    pzmap(Fdz);
    %figure(4);
    step(Fdz);
\end{lstlisting}

\begin{equation}
    \begin{split}
        V_x^{\phi_{n+1}} &= V_x^{\phi_{n}} +  \frac{ V_r\overbrace{\sum_{i}\left[ (C_i\cdot b_i)^{\phi_{n+1}} - (C_i\cdot b_i)^{\phi_{n}}\right]}^{\Delta C_i}+C_B\cdot \left(V_y^{\phi_{n+1}}-V_y^{\phi_{n}}\right)}{C_{MT}} \\
        V_x^{\phi_{n+1}} &= V_x^{\phi_{n}} + V_r \frac{ V_r \Delta C_i +C_B\cdot \left(V_y^{\phi_{n+1}}-V_y^{\phi_{n}}\right)}{C_{MT}}
    \end{split}
    \label{eq:VxPn}
\end{equation}

\begin{equation}
    \boxed{\underbrace{\int_{0}^{T}r(u)\,r\bigl(u+\tau - nT\bigr)\,du}_{v\bigl(\tau - nT\bigr)}}
\end{equation}

\begin{itemize}
    \item item 1
    
    ...
    \item item n 
\end{itemize}

\begin{enumerate}
    \item Butterworth

    \item Chebyshev
    
    \item Elliptic
    
    \item Bessel
    
    In the application in study, the group delay is a critical factor because the ECG signal is a time-domain signal, and the phase distortion can lead to a misinterpretation of the signal. So it is safe to say that the Bessel filter is the best choice for this application.
\end{enumerate}

\begin{figure}[H]   
\begin{centering}
    \begin{tikzpicture}[node distance=2cm]

        % Blocks
        \node (ntc) [block] {NTC Resistance};
        \node (ohms) [block, right of=ntc, xshift=2.5cm] {Ohms to Voltage};
        \node (mcu) [block, right of=ohms, xshift=1.5cm] {MCU};
        \node (temp) [block, right of=mcu, xshift=1cm] {Temperature};
    
        % Arrows
        \draw [arrow] (ntc) -- (ohms);
        \draw [arrow] (ohms) -- (mcu);
        \draw [arrow] (mcu) -- (temp);
    
    \end{tikzpicture}
    
    \caption{ NTC's block diagram }
    \label{fig:NTCBlock}

\end{centering}
\end{figure}
\begin{figure}[H]

    \centering
    \begin{subfigure}{0.4\textwidth}
        \includegraphics*[scale = 0.7]{Images/NovaFctHor.png}
        \caption{Nova logo horizontal}
        \label{fig:first}
    \end{subfigure}
    \hfill
    \begin{subfigure}{0.4\textwidth}
        \includegraphics*[scale = 0.05]{Images/NovaFctVer.png}
        \caption{Nova Logo vertical}
        \label{fig:second}
    \end{subfigure}
    
    \caption{Nova Logos}
    \label{fig:NovaLogos}
\end{figure}


Referece like this\textsuperscript{\cite{ESP32-datasheet}}
% Add citation commands where necessary
% Example: \cite{reference_label}